\section{CONCLUSION AND EXPECTED OUTCOMES}
\subsection{Conclusion}
In this research proposal, we analyse different methods developed in the previously related works.
And we investigate the pros and cons of these methods in order to figure out the research gap that 
we can improve in our research. Also, we provide the overview for three main attack surfaces in CI/CD
pipeline, then discussing some potential vulnerabilities and counter measure on the security of the software
supply chain. And we summarise the Top 10 risks in order to provide a clear view of which risks should
be focused more and tackle with first. 
Also, the \textbf{in-toto} framework is introduced since \textbf{SLSA Provenance} is based on in-toto format.
Since our research tool is developed based on the SLSA, we introduce key components and concepts of this 
framework as a prerequisite.
In the final section, the method and the plan of our research are being discussed. Our method tries to 
improve the method to detect suspicious updates that is not achieved in previous research.
\subsection{Expected Outcomes}
The expected outcomes of this research will be contributing some checks properly to the Macaron 
Framework. Also, finding out vulnerabilities within the popular repositories. Our work expects to 
find out if the CI/CD pipeline configuration or the protection mechanism is strong enough to 
avoid unexpected attacks. Taxonomy technique mentioned in the previous section will be 
implemented in our research to classify the discovered vulnerabilities, and finding out the primary goal of the attack according to these vulnerabilities~\cite{ohm2020backstabber}. Finally, our research will expect to 
discover the reason behind these malicious updates.   