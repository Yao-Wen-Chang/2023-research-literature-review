\section{RELATED WORK}
In the literature review, first, prerequisite definition of the CI/CD technique will be 
described. Then, a variety of risks and attacks are categorised based on the three attack 
surface within the CI/CD pipeline, which are source code repository, build and test, and deploy.
Some defense methods introduced in the related work will be aligned with the attacks in this 
section. Finally, the top ten risks existed in CI/CD pipeline referring to OWASP 2023 will be 
summarised~\cite{OWASP2023}.

\subsection{Definition of CI/CD}
CI/CD is a development and deployment process for automatically building and testing code
changes that support organizations maintain a consistent code base, then automatically shipping 
the artifact to the target customers. 'CI' involves developers frequently merging code changes 
into a central repository where automated builds and tests run. 'Build' is the process of 
converting source code into executable code. Then, running the automated tests which usually 
combined unit test and integration test against the build. These process will avoid integration 
challenges that can happen when waiting for release day to merge changes into the release branch. 
'CD' defined the process of the releases happened automatically~\cite{DoDDefCI/CD2023}. 

Nowadays, nearly all software applications were built on top of others works, which is called third-parties.
However, the convenience and capabilities of the third-party source code usually cause cybersecurity 
risks~\cite{mastrangelo2015use}. Software supply chain attacks aim at injecting code into software
components including source code base and artifacts, like docker image or executable software,
to compromise downstream users~\cite{ladisa2023sok}. 
Some risks and counter method will be introduced in the following sections.

\subsection{Three Main Stage in CI/CD}
In the following three subsections, 
we will categorise some attack events and potential attack methods into the high-level representation 
of three main categories within the CI/CD pipeline. This categorisation allows us to gain a deeper understanding of 
the security challenges and risks associated with each phase of the software development and delivery process.

\subsubsection{Source Code Repository}
Java Virtual Machine (JVM) executes Java bytecode and provides strong safety
guarantees. However, the unsafe API, "sun.misc.Unsafe", will cause serious 
security issue if it is misused by the developers. The research~\cite{mastrangelo2015use} 
studied a large repository, Maven, and analyzed the compiled Java code. The 
security issues include violating type safety, crashing the virtual machine (VM), 
uninitialized objects and so on. These misuse might impact third-party package 
management service. 

The research\cite{garrett2019detecting} proposed anomaly detection method to cluster and detect 
the suspicious update JavaScript code. It shows that this unsupervised machine learning 
clustering method is able to reduce the manual review effort by 89 percent considering the worse
case that all updates are not malicious. However, the data is skewed since more unsuspicious updates
existed in the dataset. Also, the researcher divide the features based on static method by reviewing
some packages and find out the malicious code. This method is not cost-efficient and only specific
to the programming languages where there are models already trained. 

Nevertheless, the author in~\cite{boucher2023trojan} provide an attack approach to bypass the manual
code review. With the Trojan source attacks, the attacker can embed seemingly unsuspicious Unicode into
the comments and indirectly modify the order of the code causing unexpected results. 
This would bypass the method provided by~\cite{garrett2019detecting}, 
since this feature is not recognised by the model. The author recommends the developers to include 
Regex detection method in the 'Build' stage, since this malicious code is triggered by the compiled code.
The fundamental way to avoid being compromised by the Trojan source attack is through banning the 
Unicode when it is not required in the projects. Even though recommend defense strategies 
are able to counteract this attack, but the method only deal with the specific attack against the 
CI/CD pipeline.

\subsubsection{Build/Test}
% Inject bad Dependency
% Implant CI/CD runner images and container
% Compromise CI/CD server
Malicious code injection can occur during the build of the pre-built components, especially in compiled languages.
This attack method is favored by the attackers, since the detection of the pre-built components and compiled code is 
typically more difficult~\cite{ladisa2023journey}. Therefore, opting for building package directly from 
source code is a measure to avoid this risk.

In~\cite{ohm2020backstabber}, 
an overview of the attack methods and how the attacks are triggered are being summarised. 
The most frequent way implemented to inject the bad dependencies is 'Typosquatting'. It is a simple method
to modify the packages' name with a similar one. If the consumers did not carefully check the installed 
packages, the malicious dependencies will being triggered based on the trigger point designed by the 
attackers. 
Also, this research find out the popular trigger point of the malicious dependencies is at the
installation step. At the installation step, the malicious code is usually embedded in the installation
script. When the customers install the packages, the systems are directly being compromised.  
This research paper implements the taxonomy method to deduce possible trigger points and 
the possible attacks occurring within each CI/CD stage. This method provides developers a clear 
and completed checklist to prevent the potential vulnerabilities. However, this paper does not
provide the completed counter measure against the attacks.

\subsubsection{Deploy}
% Bypass Review
% Use admin permission to add approver
In the extremely severe attack event~\cite{gentoo-incident-report}, 
the unknown entity gain access to the GitHub repository with the higher permission from the controlled
account of the repository maintainer. The accident causes malicious code to be committed and the
attacker even trying to remove various repositories. Some good practices are implemented
in this repository. For example, the audit logs function provided by GitHub enables the Gentoo organization
to respond quickly to prevent from the extra impact. However, the Gentoo GitHub organization does not
implement 2FA to prevent the accounts from being fully compromised. Also, it does not manage the identities
correctly.

\cite{vu2021lastpymile} presents a framework to detect the discrepancies of the packages on 
between source repositories and package managers. This framework can be cooperated with other tools to improve
efficient and effective. However, the research does not consider the situation that there are no
discrepancies between the source code and the published packages, but the code have already been maliciously injected.
In the framework we deploy in this research, the provenance can ensure there are no suspicious contributors contributing
to the code base. 

\subsection{OWASP Top 10 CI/CD Risks\cite{OWASP2023}}
\begin{enumerate}[label=(\arabic*)]
    % 1
    \item \textbf{Insufficient Flow Control Mechanisms}
    
    \textbf{Definition: }The attacker successfully gained unauthorised access to a critical system within the CI/CD process. 
    However, it became apparent that the compromised system lacked adequate enforcement mechanisms to rigorously 
    approve and review the code or artifacts provided by contributors, leaving a significant gap in the security infrastructure.

    \textbf{Impact: }
        \begin{itemize}
            \item The attackers can sneakily add malicious code to a repository branch. 
            This code can either be automatically deployed to the production system or activated manually by the attackers themselves.
            \item Upload an artifact to the artifact repository, such as a package or container, 
            in the guise of a legitimate artifact created by the build environment and picked 
            up by a deployment pipeline and deployed to production.
        \end{itemize}
    
    \textbf{Remediation:}
    \begin{itemize}
        \item Configure strict branch protection rules
        \item Limit the usage of auto-merge rules
        \item Prevent drifts and inconsistencies between the running code in production and its 
        CI/CD origin.
    \end{itemize}
    % 2
    \item \textbf{Inadequate identity and Access Management}
    
    \textbf{Definition: }This risks stem from the difficulties in managing the vast amount of
    identities. The identities are identified through personal access token, e-mail, password and so on.
        \begin{itemize}
            \item Overly permissive identities
            \item Stale identities - Some identities that are not active or no longer require access but have
            not had their account deactivated.
            \item External identities - (1) Employees registered with email from a domain not managed by 
            the organization (2) External collaborators are outside the organization's control.
        \end{itemize}

    \textbf{Impact: }
        \begin{itemize}
            \item Overly permissive accounts leads to a state where the attacker can compromise any user
            account on any system within the CI/CD pipeline.
        \end{itemize}

    \textbf{Remediation:}
        \begin{itemize}
            \item Consistently checked and connected the accounts of individuals with their access rights, 
            and eliminated any unnecessary access rights that weren't needed for their current tasks.
            \item Ensure the identities are aligned to the principle of the least privilege, and pre-defined an expiry date 
            for the identities' permissions.
            \item Prevent the employees from using personal email addresses.
            \item Avoid the shared accounts. Created the dedicated accounts for each specific context.
        \end{itemize}
    % 3
    \item \textbf{Dependency Chain Abuse}

    \textbf{Definition: }
        Dependency chain abuse refer to an attacker's ability to abuse flaws relating to how 
        engineering workstations and build environments fetch code dependencies. The build system downloads the 
        malicious package instead of the one intended to be pull. There are four scenarios where the developers might be tricked.
        \begin{itemize}
            \item Dependency confusion - Publication of malicious packages in public repositories with the same 
            names as those private one.
            \item Obtain the control of the account of the package maintainer in order to upload the malicious version.
            \item Typosquatting - Publication of similar names to those popular packages.
            \item Brandjacking - The malicious packages were consistent with the naming convention with the trusted brand.
        \end{itemize}

    \textbf{Impact: }
        Once the malicious code is running, it can be leveraged for credentials theft and move horizontally through a system and 
        network.

    \textbf{Remediation:}
        \begin{itemize}
            \item Ensure the packages are not directly pulled through the internet, 
            but through an internal proxy. And disallow pulling directly from external repositories.
            \item Verify checksum and signature of the pulled packages.
            \item Lock the packages' version instead of pulling the latest version.
            \item Installation scripts should not access to 
            sensitive resources in the build process.
            \item Always ensure projects contain configuration files of package managers.
            \item The most important is deployment of a quick detection, 
            monitoring and mitigation to avoid further compromise.
        \end{itemize}
    % 4
    \item \textbf{Poisoned Pipeline Execution (PPE)}

    \textbf{Definition: }
        The attacker access to the source control systems, but without access to the build environment, is
        able to manipulate the build process by injecting malicious code into the build configuration file.
        There are three type of PPE, direct PPE (D-PPE), indirect PPE (I-PPE) and public-PPE (3PE).

        In the D-PPE scenario, the attackers modify the CI config files either by submitting a PR or directly pushing to the unprotected
        remote branch. Since the CI pipeline execution is triggered by push or PR events, and the CI execution
        is defined by CI Configuration file, the malicious commands run in the build node.

        In the I-PPE scenario, the pipeline is configured to pull the CI configuration file from a protected 
        branch or CI build is defined by the CI system instead of the in the file stored in the source code.
        In those cases, the attackers can still inject malicious code into the files referenced by the pipeline
        configuration file.

        In most cases, the permissions of the access to the repository are given to the organization
        members. However, in the 3PE scenario, the public repositories are allowed the anonymous to 
        contribute. If the CI pipeline runs unreviewed code, the repository is susceptible to the 3PE.

    \textbf{Impact: }
        \begin{itemize}
            \item Access to the secret available to the CI job.
            \item Able to ship code and artifacts further down the pipeline, in the guise of legitimate 
            code build by the build process.
        \end{itemize}

    \textbf{Remediation:}
        \begin{itemize}
            \item Ensure that pipelines running unreviewed code are executed on isolated nodes to prevent
            exposure of sensitive information.
            \item To prevent the manipulation of the CI configuration file.
            \item Remove permissions from the users that do not need them.
        \end{itemize}
    % 5
    \item \textbf{Insufficient PBAC (Pipeline-Based Access Controls)}

    \textbf{Definition: }
        Adversary is able to execute malicious code within the context of the pipeline.
        The pipeline execution nodes have access to the resources or systems within and outside the 
        execution environment.

    \textbf{Impact: }
        Malicious code is able to run in the context of the pipeline. Probability, this attack would lead to 
        the exposure of the secret and confidential data.
        
    \textbf{Remediation:}
        \begin{itemize}
            \item Do not use shared node for pipeline with different levels of sensitivity.
            \item Where applicable, run pipeline jobs on a separate, dedicated node.
        \end{itemize}
    % 6
    \item \textbf{Insufficient Credential Hygiene}

    \textbf{Definition: }
        CI/CD environments are built of multiple systems communicating and authenticating against each other
        through verifying the credentials. Insufficient credential Hygiene generally means the overly permissive
        or the credentials are accidentally existed in CI/CD pipelines and the code repositories. Some cases are
        due to the unrotated credentials issue.  

    \textbf{Impact: }
        The adversary obtains the credentials to deploy the malicious code and artifacts.
    \textbf{Remediation:}
        \begin{itemize}
            \item Conform to the least privilege rule and granted the exact set of permission.
            \item Avoid sharing the same sets of credentials across multiple contexts.
            \item Using temporary credentials. If using static credentials, better periodically rotate
            all the static credentials and detect stale credentials.
            \item Scoping the credentials to specific source IP to ensure the credentials cannot be used
            outside the environment even if it is compromised.
            \item Detect secrets pushed to the code repositories.
        \end{itemize}
    % 7
    \item \textbf{Insecure System Configuration}

    \textbf{Definition: }
        Flaws in the security settings and configuration, which often results in easily compromised by The
        attackers. For example, the overly permissive network access controls allow the attackers to interact
        with different domain.

    \textbf{Impact: }
        The flaws may be abused by the attacker to manipulate CI/CD flows, and obtain the sensitive tokens.
    
    \textbf{Remediation:}
        \begin{itemize}
            \item Ensure the network access the systems is aligned with the principle of the least access.
            \item Periodically review all system configuration.
            \item Ensure permission to pipeline execution nodes follow the least privilege principle. 
        \end{itemize}
    % 8
    \item \textbf{Ungoverned Usage of 3rd Party Services}

    \textbf{Definition: }   
        3rd party services are granted access to the organization's assets, such as the CI/CD systems.

    \textbf{Impact: }
        Lack of governance and visibility around 3rd part might allow the write permission on the repository.
        Then, the flaw is leveraged by the adversary to push the code to the repository.
    \textbf{Remediation:}
        \begin{itemize}
            \item Define the scoped context that the 3rd parties are able to access, and with strict ingress
            and egress filter.
            \item Established vetting procedures to verify the trustworthiness of the 3rd parties. Prior to being 
            granted access to the environment, the approval of being granted to resources should be verified.
        \end{itemize}
    % 9
    \item \textbf{Improper Artifact Integrity Validation}

    \textbf{Definition: }
        This flaw allows an attacker with access to the CI/CD process to push malicious code or artifacts down 
        the pipeline.

    \textbf{Impact: }
        Execution of malicious code.

    \textbf{Remediation:}
        \begin{itemize}
            \item Validate the integrity of resources all the way from development to production.
            \item Code signing to prevent unsigned commits from flowing down the pipeline.
            \item Prior to fetching and using 3rd parties, the hash of the 3rd parties should be calculated and 
            cross-referenced against the official published hash of the resource provider.
        \end{itemize}
        
    % 10
    \item \textbf{Insufficient logging and visibility}

    \textbf{Definition: }
        The risk allow the adversary to carry out malicious activities without being detected.
        For example, the permission modification and execution of builds.

    \textbf{Impact: }
        Fail to detect a breach may face difficulties in mitigation. Time and data are the most 
        valuable commodities to an organization under the attack.

    \textbf{Remediation:}
        \begin{itemize}
            \item First, be familiar with different systems involved in the potential threats.
            \item Make sure all relevant logs are enables.
            \item Shipping logs to a centralized location and creating the alerts to detect malicious activities.
        \end{itemize}
\end{enumerate}

\subsection{How SLSA Framework Solve Frequent Risks in CI/CD}
In this section, we will provide some examples to show how SLSA is able to mitigate the risks listed in OWASP Top10.
SLSA specifies a list of requirements which will enforce secure practice against the most frequent risks.
In the \textbf{Two-person Reviewed}, it requires the pull request being reviewed by another authentic reviewer.
The method will make \textbf{Insufficient Flow Control Mechanisms} risk impossible.
Also, \textbf{Dependencies Complete} is to make sure the SLSA provenance records all dependencies during
build step. Then, the developers can trust the provenance and inspect all dependencies. This requirement 
will indirectly defense \textbf{Dependency Chain Abuse}.



\subsection{ U.S. Department of Defense Recommend \cite{DoDDefCI/CD2023}}
\begin{enumerate}
    \item \textbf{Zero Trust Approach}
    
        No user, endpoint device or process is fully trusted.

    \item \textbf{Strong Cryptographic Algorithm}
    
        Avoid using outdated cryptographic algorithm which poses a threat to CI/CD pipelines.
        The threat includes sensitive data exposure and keys generated across the CI/CD 
        pipeline.

    \item \textbf{Minimize the Use of Long-Term Credentials}

    \item \textbf{Add Signature to CI/CD Configuration and Certify It}

        Ensure the code change is continuously signed, and the signature is verified throughout
        CI/CD process. If the signing identity itself is compromised, it undermines trust.
    
    \item \textbf{Two-Person Rules for all Code Updates}

        The developer checks in the code which should be reviewed and approved by another 
        developer. 

\end{enumerate}

\subsection{in-toto Framework}
in-toto is an end-to-end security framework aim to protect the software supply chain. It provides a variety
of mechanism for each steps in the CI/CD to check the integrity of the previous steps. Cryptography and hash 
function are implemented in in-toto.

\textbf{Layout} is a file format to document the actors, timestamp and actions with each step of the supply chain.
It provides the information for downstream steps to verify. This information is the cryptographic hash of action, actors and some other security
related data. Therefore, the downstream steps can verify the integrity based on the hash. This is the high level structure 
compared to Link Metadata.

\textbf{Link MetaData} is a lower level structure compared to \textbf{Layout}, which provides much detailed information about each action.

\textbf{SLSA} framework adopted in our research is based on the in-toto framework.

\subsection{Conclusion}
Despite the previously introduced methods seems to address all the security issues
existed in the code base and within the CI/CD, some of them may overemphasize one 
particular approach to address software supply chain security without considering 
compounding factors that impact risk.
Some projects aim at providing single solution that conflates multiple 
objectives~\cite{melara2022software}.
For instance, SAP's Risk Explorer is designed to cultivate a community of users 
and contributors to the taxonomy. Users can explore the taxonomy by collapsing and 
expanding nodes in an attack tree. Detailed information about attack vectors, 
references, and safeguards is presented beneath the tree. 
This tool addresses all potential risks associated with an entire CI/CD pipeline, 
as referenced in~\cite{ladisa2023journey}.

Similarly, the in-toto framework is dedicated to ensuring the integrity of the supply 
chain pipeline while harmonizing multiple objectives.

The framework employed in this research is based on Supply Chain Level Security 
Artifacts (SLSA). In the following section, we will introduce and discuss Supply Chain 
Level Security Artifacts (SLSA) in more detail.
