\section{Defense Method}
% provide method here

These goals are seek to establish the trust in the software supply chain by verifying in-
formation about the participants or processes \cite{melara2022software}.
The complex CI/CD has the potential to catch the attackers more quickly via peer review, 
because the developers and distributors get a chance to review the code, increasing
the chances of malicious code being discovered \cite{levy2003poisoning}.
Providing cryptographic hashes if the software packages is of significance to
verify the software's integrity \cite{levy2003poisoning}.
Signing the releases of the packages by the providers with the public-key enable the 
users to verify the 
\subsection{ U.S. Department of Defense Recommend \cite{DoDDefCI/CD2023}}
\begin{enumerate}
    \item \textbf{Zero Trust Approach} \\
        No user, endpoint device or process is fully trusted.
    \item \textbf{Strong Cryptographic Algorithm} \\
        Avoid using outdated crytographic algorithm which poses a threat to CI/CD pipelines.
        The threat includes sensitive data exposure and keys generated across the CI/CD 
        pipeline. 
    \item \textbf{Minimize the Use of Long-Term Credentials}
    \item \textbf{Add Signature to CI/CD Configuration and Cerify It} \\
        Ensure the code change is continuously signed, and the signaature is verified throughout
        CI/CD process. If the signing identity itself is compromised, it undermines trust.
    \item \textbf{Two-Person Rules for all Code Updates}    
        The developer checks in the code which should be reviewed and approved by another 
        developer. 

\end{enumerate}
Some of the projects aims at providing single solution that conflates multiple objectives 
\cite{melara2022software}.

Despite the previously introduced methods seems to address all the security issue
existed in the code base and within the CI/CD, some of them may overemphasize one 
particular approach to address software supply chain security. without considering 
compounding factors that impact risk \cite{melara2022software}.