\section{Research Plan}


\begin{table}[h]
  \centering
  \begin{adjustbox}{max width=\columnwidth}
  \begin{tabular}{|p{3.2cm}|p{2.6cm}|p{2.6cm}|}
      \hline
      \textbf{Research Stage} & \textbf{Deadline} & \textbf{Challenges} \\
      \hline
      Data Collection & Month 1 & Finding relevant sources \\
      \hline
      Examine Repositories & Month 2 & Defining research questions \\
      \hline

  \end{tabular}
  \end{adjustbox}
\end{table}



The project will involve three phases:


1. Defining Unsafe Updates: Building on similar research conducted in the JavaScript 
ecosystem (https://ieeexplore.ieee.org/document/8805698), the project's first phase 
involves defining what an unsafe update means within the context of Python and/or Java. 
Typically, an unsafe update could be one that introduces breaking changes, 
negatively affects performance, opens up security vulnerabilities, 
or adds incompatible API changes.


2. Implementation of Safety Checks: Next, we will extend the Macaron framework's 
functionality by implementing additional safety checks for these unsafe updates.
Macaron is an extensible checker framework for supply chain security and CI/CD services,
such as GitHub Actions. It allows adding new checks as Python modules and provides 
intermediate representations specifically designed for CI/CD services to facilitate 
verifying new properties.


3. Empirical Analysis of Real-World Projects: With the safety checks in place, 
the final phase of the project is an empirical study conducted on GitHub to ascertain
the frequency of unsafe updates occurring in Python and/or Java projects.
By understanding the 'how' and 'why' behind these updates, developers can adopt more 
informed, proactive strategies in their coding practices.