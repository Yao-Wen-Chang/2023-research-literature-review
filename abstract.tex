\begin{abstract}
The popularity of the open-source package surged, attracting more and more projects that are prone 
to deploying 3rd-party artifacts.
On the other hand, the increasing popularity of automatic CI/CD systems supports developers in 
building their projects at an extremely fast pace.
However, this convenience and the addition of new features increase the attack surface, 
providing attackers with extra ways to target downstream users

This literature review will introduce Continuous Integration/Continuous Delivery (CI/CD).
Also, potential attack surfaces and how the malicious attackers exploit these attack surfaces
to compromise CI/CD pipelines will be covered in this review. Multiple methods and 
frameworks are going to be introduced to counter the attack. 
Some method target risks at certain stage in the pipeline, and some cover the whole pipeline.

In our research, we are going to adopt the framework, Macaron, which is based on the Supply Chain Level Security Artifacts 
(SLSA) framework. We will design our method to fetch 3rd party repository from GitHub as our research data.  
\end{abstract}


