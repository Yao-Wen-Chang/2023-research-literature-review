\begin{abstract}
  The popularity of open-source packages has experienced a significant surge, attracting an ever-growing number of projects that rely on the deployment of third-party artifacts. 
  Simultaneously, the rising prevalence of automatic Continuous Integration/Continuous Delivery (CI/CD) systems has empowered developers to accelerate their project development 
  processes dramatically. However, this convenience, coupled with the continuous addition of new features, has expanded the attack surface, offering malicious actors additional 
  avenues to target downstream users.

  This literature review that we are going to discuss in the related work section serves as an introduction to the world of Continuous Integration/Continuous Delivery (CI/CD). 
  It delves into the potential attack surfaces that exist within CI/CD pipelines and explores how malicious attackers 
  exploit these vulnerabilities to compromise these critical systems. Throughout this review, we will introduce various 
  methods and frameworks designed to mitigate these threats. Some of these methods target specific risks at particular 
  stages in the pipeline, while others provide comprehensive coverage across the entire CI/CD process.
  
  In our research, we will contribute to Macaron framework and further to adopt it to analyse the vulnerabilities in the repositories during
  the software supply chain. This framework is built upon the principles of the Supply 
  Chain Level Security Artifacts (SLSA) framework. We will design our research methodology to involve the retrieval of 
  third-party repositories from GitHub as our primary source of research data, analysis through Macaron, and scanning source code through downstream automation scanner.
  At the end of the research, we will evaluate the results through the metric we defined in this research proposal. 

  Our research aims to contributor a framework for scrutinizing artifacts and ensuring their safe passage 
  through the supply chain. We will also provide the scanning results and engage in discussions 
  regarding the implementation of best practices and the development of a secure software supply 
  process.  
\end{abstract}
