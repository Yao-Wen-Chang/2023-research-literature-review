\section{Introduction}
CI/CD is a development method used to efficiently build and test code updates. 
It helps organizations keep their software consistent and allows them to smoothly 
incorporate new changes. However, CI/CD systems can be tempting targets for cyber attackers. 
These attackers may try to insert malicious code into CI/CD processes, 
steal valuable credentials, or disrupt the original functioning of applications.

Recent incidence like the infection of SolarWind's Orien platform \cite{ladisa2023sok, 
peisert2021perspectives} which is used to monitor and manage the network is downloaded by 
thousands of customers, including U.S. government agencies, critical infrastrure providers, 
and private companies. 

Another malicious attack target the credentials from a contributor of the esline-scope package.
The attackers updated the malicious code within the code base, which will end up stealing
multiple credentials from the downstream users \cite{eslint2018}.

Section 2 will briefly introduce CI/CD. Section 3 would target the attack
surface within the process of CI/CD and the counter method. In section 4, our literature 
review would introduce SLSA framework from Google, and explain how SLSA can patch 
the vulnerable CI/CD process. In section 6, the aim and objects of the research 
will be explained. And the research plan will be introduced in section 7. 


