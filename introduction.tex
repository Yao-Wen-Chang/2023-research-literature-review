\section{Introduction}
CI/CD is a development method used to efficiently build and test code updates. 
It helps organizations keep their software consistent and allows them to smoothly 
incorporate new changes. However, CI/CD systems can be tempting targets for cyber attackers. 
These attackers may try to insert malicious code into CI/CD processes, 
steal valuable credentials, or disrupt the original functioning of applications.

Recent incidence like the infection of SolarWind's Orien platform \cite{ladisa2023sok, 
peisert2021perspectives} which is used to monitor and manage the network is downloaded by 
thousands of customers, including U.S. government agencies, critical infrastrure providers, 
and private companies. 

Another malicious attack target the credentials from a contributor of the esline-scope package.
The attackers updated the malicious code within the code base, which will end up stealing
multiple credentials from the downstream users \cite{eslint2018}.

In Section 2, we delve into the related work on the CI/CD pipeline security. Three main 
attack surfaces within CI/CD pipeline and the counter measures are introduced here. 
Additionally, we will summarise the OWASP Top 10 CI/CD Risks. These will provide a comprehensive
foundation for our research. 

In Section 3, we will explore the realm of Software Supply Chain Security, which is the prerequisite
for understanding the key components of the Macaron Framework implemented in our research.
We dissect the Software Supply Chain, emphasizing the critical importance of provenance. 
Also, we will explain the nuanced difference between Software Bill of Materials (SBOM) and the SLSA Provenance.
Finally, we investigate the intricacies of the Build Model and the Security Levels specified by the SLSA framework.

In Section 4, we delineate our research aims and objectives. Our overarching aim is to fortify the security aspects of CI/CD pipelines. 
And our objectives include how we are going to achieve our aims. Furthermore, the current research 
progress and the research gap we find out in related works will be improved by our contribution.

Section 5 provides a detailed timetable and plan for the research project. 
We outline the key milestones, tasks, and deadlines necessary for successful project completion.

In Section 6, we offer a concise conclusion to the research proposal, summarizing our intentions and expectations.