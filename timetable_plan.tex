\section{TIMETABLE AND PLAN}
We split the research plan into three phases, also, providing a timetable for understanding 
the research schedule.

\subsection{Phase One}
Defining Unsafe Updates: Building on similar research conducted in the JavaScript ecosystem 
\\(https://ieeexplore.ieee.org/document/8805698), the project's first phase 
involves defining what an unsafe update means within the context of Python and/or Java. 
Typically, an unsafe update could be one that introduces breaking changes, 
negatively affects performance, opens up security vulnerabilities, 
or adds incompatible API changes.

In order to discover different type of unsafe updates and the target victim, we will review 
related work, articles, and CI/CD attack events to support the understanding of the CI/CD 
attacks existed nowadays. 

\subsection{Phase Two}
Implementation of Safety Checks: Next, we will extend the Macaron framework's 
functionality by implementing additional safety checks for these unsafe updates.
Macaron is an extensible checker framework for supply chain security and CI/CD services,
such as GitHub Actions. It allows adding new checks as Python modules and provides 
intermediate representations specifically designed for CI/CD services to facilitate 
verifying new properties. We will begin from implementing SLSA Level 4 check including Two 
Person Review, Verified History, and Retained indefinitely.

A malicious and suspicious event on GitHub is conducted by one user account "pastramahodu"
which forked nearly 2000 repositories. The researcher find out the attacker try to use script
to automatically fetch and detect the target file type, such as package.json, sh, and so on.
Then, the script creating pull request against each victim repositories to compromise the 
repositories that are not configured properly. The malicious code injection will be triggered
in CI pipelines to exfiltrate repository name and hostname back to webhook server~\cite{gelb2023mass}.
Therefore, we decide to implement the following three checks to avoid the malicious code being 
merged into the branch.

The Two Person Review check ensures each pull request is reviewed by at least one authentic reviewer.
This check deal with the situation that someone wants to bypass the review and directly merge 
their code into the branch. Our method will fetch all the pull requests from the branch 
specified by the users of Macaron Framework.

The Verified History ensures at least one strong authenticated participant of the revision's history.
The identities should be authenticated through two-step verification. The first step usually 
verified the password, and second step might be verified through SMS or Email. This check grapple
with the situation when the identities' account are being compromised.

Retained indefinitely will check if the commits are preserved for 18 months; therefore, the
consumers can trust the artifacts they are going to use in their applications are not being 
modified by suspicious contributors.

The remaining session, we will keep contributing to the framework, but only implementing a
specific version on the phase three.
\subsection{Phase Three}
Empirical Analysis of Real-World Projects: With the safety checks in place, 
the final phase of the project is an empirical study conducted on GitHub to ascertain
the frequency of unsafe updates occurring in Python and/or Java projects.
By understanding the 'how' and 'why' behind these updates, developers can adopt more 
informed, proactive strategies in their coding practices. We will design a script to fetch
our data, which will be designed to perform the Python script Listing~\ref{pythoncode}.

Furthermore, we will build graph and even attack tree to classify our finding and visualize 
the result. In this way, the developers can easily understand what they should focus 
and improve in their current configuration of the CI/CD pipeline or the future projects.

\lstdefinestyle{mystyle}{
    language=Python,
    basicstyle=\small\ttfamily,
    commentstyle=\color{gray},
    keywordstyle=\color{blue},
    numbers=left,
    numberstyle=\tiny\color{gray},
    frame=single,
    breaklines=true,
    breakatwhitespace=true,
    tabsize=4
}
\begin{lstlisting}[style=mystyle, caption={Fetch Top 100 Java and Python Repositories}, label=pythoncode]
  import requests
  import os
  import sys
  access_token = os.getenv("GITHUB_TOKEN")
  if access_token is None:
    print(f"Warning: {access_token} environment variable is not set.")
    sys.exit(1)    
  url = "https://api.github.com/search/repositories"  
  language = ['java', 'python']
  params = {
      'q': f'language:{language}',  # Filter for repositories with more than 0 stars
      'sort': 'stars',  # Sort by stars
      'order': 'desc',  # Sort in descending order (most stars first)
      'per_page': 100,  # Number of results per page (max is 100)
      'page': 1,  # Page number (start with 1)
  }
  headers = {
      'Authorization': f'token {access_token}',
  }
  # Send the GET request to the GitHub API
  response = requests.get(url, params=params, headers=headers)

  # Check if the request was successful
  if response.status_code == 200:
      # Parse the JSON response
      data = response.json()

      # Print information about the top repositories
      for idx, repo in enumerate(data['items'], start=1):
          print(f"{idx}. {repo['name']} - {repo['html_url']} ({repo['stargazers_count']} stars)")
  else:
      print(f"Error: {response.status_code} - {response.text}")
\end{lstlisting}

\subsection{Timeline}
\newganttchartelement{orangebar}{
    orangebar/.style={
        inner sep=0pt,
        draw=red!66!black,
        very thick,
        top color=white,
        bottom color=orange!80
    },
    orangebar label font=\slshape,
    orangebar left shift=.1,
    orangebar right shift=-.1
}

\newganttchartelement{bluebar}{
    bluebar/.style={
        inner sep=0pt,
        draw=purple!44!black,
        very thick,
        top color=white,
        bottom color=blue!80
    },
    bluebar label font=\slshape,
    bluebar left shift=.1,
    bluebar right shift=-.1
}

\newganttchartelement{greenbar}{
    greenbar/.style={
        inner sep=0pt,
        draw=green!50!black,
        very thick,
        top color=white,
        bottom color=green!80
    },
    greenbar label font=\slshape,
    greenbar left shift=.1,
    greenbar right shift=-.1
}

\begin{center}
\resizebox{\textwidth}{!}{ % Resize the chart to fit the page width
\begin{ganttchart}[
  hgrid style/.style={black, dotted},
  vgrid={*5{black,dotted}},
  x unit=9mm,
  y unit chart=9mm,
  y unit title=12mm,
  time slot format=isodate,
  time slot unit=month, 
  % group incomplete/.append style={fill=groupblue},
  group label font=\bfseries \Large,
  % group progress label font=\bfseries\small,
  bar progress label node/.append style={right=7pt},
  group progress label node/.append style={right=7pt},
]{2023-7-24}{2024-5-31}
\gantttitlecalendar{year, month} \\ 
\ganttgroup[
  group/.append style={fill=orange}
]{Phase One}{2023-7-31}{2023-8-29}\\
\ganttorangebar{Literature Review}{2023-7-31}{2023-8-29}\\
\ganttorangebar{Define Unsafe Updates}{2023-7-31}{2023-8-29}\\
\ganttgroup[
  group/.append style={fill=blue}
]{Phase Two}{2023-7-31}{2024-5-31}\\
\ganttbluebar[bar height=0.5]{Implement Two Person Review Check}{2023-7-31}{2023-9-12}\\
\ganttbluebar[bar height=0.5]{Implement Verified History Check}{2023-8-27}{2023-9-30}\\
\ganttbluebar[bar height=0.5]{Implement Retained indefinitely}{2023-9-27}{2023-10-12}\\
\ganttbluebar[bar height=0.5]{Implement Other Functions}{2023-10-12}{2024-5-31}\\

\ganttgroup[
  group/.append style={fill=green}
]{Phase Three}{2023-9-18}{2024-5-31}\\
\ganttgreenbar{Collect Data}{2023-9-18}{2024-1-23}\\
\ganttgreenbar{Empirical Analysis of Real-World Projects}{2024-1-23}{2024-5-31}\\
\end{ganttchart}
}
\end{center}